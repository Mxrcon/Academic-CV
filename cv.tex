%------------------------------------
% Dario Taraborelli
% Typesetting your academic CV in LaTeX
%
% URL: http://nitens.org/taraborelli/cvtex
% DISCLAIMER: This template is provided for free and without any guarantee 
% that it will correctly compile on your system if you have a non-standard  
% configuration.
% Some rights reserved: http://creativecommons.org/licenses/by-sa/3.0/
%------------------------------------

%!TEX TS-program = xelatex
%!TEX encoding = UTF-8 Unicode

\documentclass[10pt, a4paper]{article}
\usepackage{fontspec} 

% DOCUMENT LAYOUT
\usepackage{geometry} 
\geometry{a4paper, textwidth=5.5in, textheight=8.5in, marginparsep=7pt, marginparwidth=.75in}
\setlength\parindent{0in}

% FONTS
\usepackage[usenames,dvipsnames]{xcolor}
\usepackage{xunicode}
\usepackage{xltxtra}
\defaultfontfeatures{Mapping=tex-text}
%\setromanfont [Ligatures={Common}, Numbers={OldStyle}, Variant=01]{Linux Libertine O}
%\setmonofont[Scale=0.8]{Monaco}
%%% modified by Karol Kozioł for ShareLaTeX use
\setmainfont[
  Ligatures={Common}, Numbers={OldStyle}, Variant=01,
  BoldFont=LinLibertine_RB.otf,
  ItalicFont=LinLibertine_RI.otf,
  BoldItalicFont=LinLibertine_RBI.otf
]{LinLibertine_R.otf}
\setmonofont[Scale=0.8]{DejaVuSansMono.ttf}

% ---- CUSTOM COMMANDS
\chardef\&="E050
\newcommand{\html}[1]{\href{#1}{\scriptsize\textsc{[html]}}}
\newcommand{\pdf}[1]{\href{#1}{\scriptsize\textsc{[pdf]}}}
\newcommand{\doi}[1]{\href{#1}{\scriptsize\textsc{[doi]}}}
% ---- MARGIN YEARS
\usepackage{marginnote}
\newcommand{\amper{}}{\chardef\amper="E0BD }
\newcommand{\years}[1]{\marginnote{\normalsize #1}}
\renewcommand*{\raggedleftmarginnote}{}
\setlength{\marginparsep}{7pt}
\reversemarginpar

% HEADINGS
\usepackage{sectsty} 
\usepackage[normalem]{ulem} 
\sectionfont{\mdseries\upshape\Large}
\subsectionfont{\mdseries\scshape\normalsize} 
\subsubsectionfont{\mdseries\upshape\large} 

% PDF SETUP
% ---- FILL IN HERE THE DOC TITLE AND AUTHOR
\usepackage[%dvipdfm, 
bookmarks, colorlinks, breaklinks, 
% ---- FILL IN HERE THE TITLE AND AUTHOR
	pdftitle={Glassman CV},
	pdfauthor={Elena Leah Glassman},
	pdfproducer={http://nitens.org/taraborelli/cvtex}
]{hyperref}  
\hypersetup{linkcolor=blue,citecolor=blue,filecolor=black,urlcolor=MidnightBlue} 

% DOCUMENT
\begin{document}
{\LARGE Elena Leah Glassman}\\[0.5cm]
University of California, Berkeley\\
%354 Hearst Memorial Mining Building\\
Berkeley, CA USA\\%[.2cm] \texttt{94720} 
\texttt{+1 215-694-9631}\\
 %\\%[.2cm]
\href{mailto:glassman@alum.mit.edu}{glassman@alum.mit.edu} \\
\href{http://eglassman.github.io/}{eglassman.github.io}\\ 

%\vfill
%Born:  March 12, 1879---Ulm, Germany\\
%Nationality:  German/American

%%\hrule
% \section*{Current position}
% \emph{Postdoctoral Scholar}, Berkeley Institute of Design, UC Berkeley EECS

%%\hrule
\section*{Areas of specialization}
Human-computer interaction • Programming education at scale • Program synthesis using examples

%%\hrule
\section*{Academic positions}
\noindent
\years{2016-present}Postdoctoral Scholar\hfill Berkeley Institute of Design, EECS, UC Berkeley\\
\years{2012-2016}Graduate researcher\hfill User Interface Design Group, CS \& AI Lab, MIT\\
\years{2010-2011}Visiting researcher\hfill Biomimetics \& Dexterous Manipulation Lab, Stanford University\\
\years{2008-2011}Graduate researcher\hfill Robot Locomotion Group, CS \& AI Lab, MIT\\
\years{2004-2008}Undergraduate researcher\hfill CS \& AI Lab, MIT\\
\years{2003-2004}Volunteer researcher\hfill EEG Lab, Princeton University

%%\hrule
\section*{Industry positions}
\noindent
\years{2016}Research scientist (contractor) \hfill Search, Google\\
\years{2015}User experience research intern \hfill Search, Google\\
\years{2014}Design research intern \hfill neXus Research Team, Microsoft Research\\

%\hrule
\section*{Education}
\noindent
\years{2016}\textsc{Ph.D.} in Electrical Engineering \& Computer Science\hfill MIT\\
\years{2010}\textsc{M.Eng.} in Electrical Engineering \& Computer Science\hfill MIT\\
\years{2008}\textsc{B.S.} in Electrical Science \& Engineering\hfill MIT\\

\section*{Selected fellowships and scholarships}
\noindent
\years{2014}MIT Amar Bose Teaching Fellow, for developing innovative tools for teaching CS at scale\\
\years{2011-2014}NSF Graduate Research Fellow (NSF GRFP)\\
\years{2008-2011}National Defense Science and Engineering Graduate Fellow (NDSEG)\\
\years{2004} IEEE President's Scholarship (\$10,000)\\


%\hrule
\section*{Selected honors \& awards}
\noindent
%\years{2015}Invited research presentation, Rising Stars workshop for aspiring CS faculty, MIT
\years{2016}Audience Choice Award, MIT Can Talk speech competition\\
\years{2015}Accepted into Rising Stars workshop for aspiring CS faculty\\
\years{2009}Masterworks Oral Thesis Presentation Award, MIT EECS\\
\years{2008}Inducted into Eta Kappa Nu, EECS Honor Society\\
\years{2004}Valedictorian \& commencement speaker, Central Bucks High School West\\
\years{2004}Inducted into the National Gallery for America's Young Inventors\\
\years{2003} Intel International Science and Engineering Fair -- Best of Category: Computer Science (\$5,000)\\
\years{2003} Intel Foundation Young Scientist Award (\$50,000) \\ \emph{Awarded to the top 3 individual projects at Intel International Science \& Engineering Fair}

\section*{Teaching}

\subsection*{Experience}

\years{2016}Co-lecturer, User Interface Design \& Implementation ($\approx$ 175 students) \hfill MIT EECS\\
\years{2013}Co-lecturer, introductory python programming \hfill MIT MEET, Jerusalem\\
\years{2013}Educational video script writer, radio receiver technology \hfill MIT Teaching \& Learning Lab\\
\years{2012-2014}Teaching assistant, Computation Structures\hfill MIT EECS\\
\years{2011}Teaching assistant, Introduction to EECS 1\hfill MIT EECS\\
\years{2006-2011}Tutor, Signals, Systems, \& Probabilistic Systems Analysis\hfill MIT EECS Honor Society

\subsection*{Certifications}

\years{2011}Graduate Student Teaching Certificate\hfill MIT Teaching \& Learning Lab\\

\section*{Human-Computer Interaction Publications}

\subsection*{Journal articles}
\noindent
%\emph{Learning at scale}\\
\years{2015 TOCHI}\textbf{EL Glassman}, J Scott, R Singh, P Guo, RC Miller.\\ “OverCode: visualizing variation in student solutions to programming problems at scale."\\ \emph{ACM Transactions on Computer-Human Interaction}, 22 (2).
%\emph{Biomedical signal processing}\\
%\years{2005}\textbf{EL Glassman}. “A wavelet-like filter based on neuron action potentials for analysis of human scalp electroencephalographs." \emph{IEEE Transactions on Biomedical Engineering (TBME)} 52 (11), 1851-1862.\\

\subsection*{Conference papers}
\noindent
%\emph{Learning at scale}\\
\years{2017 L@S}A Head, \textbf{EL Glassman}, G Soares, R Suzuki, L Figueredo, L D'Antoni and B Hartmann.\\ ``Writing Reusable Code Feedback at Scale with Mixed-Initiative Program Synthesis.''\\ \emph{ACM Learning at Scale}.\\
\years{2016 ASIST}\textbf{EL Glassman}, DM Russell.\\ ``DocMatrix: Self-Teaching from Multiple Sources.''\\ ASIS\&T Annual Meeting.\\
\years{2016 CSCW}\textbf{EL Glassman}, A Lin, CJ Cai, RC Miller.\\ “Learnersourcing Personalized Hints."\\ \emph{ACM Computer-Supported Cooperative Work and Social Computing}.\\
\years{2015 UIST}\textbf{EL Glassman}, L Fischer, J Scott, RC Miller.\\ “Foobaz: Variable Name Feedback for Student Code at Scale."\\ \emph{ACM Symposium on User Interface Software \& Technology}.\\
\years{2015 CHI} \textbf{(Best of CHI Honorable Mention)}\\ \textbf{EL Glassman}, J Kim, A Monroy-Hernández, MR Morris.\\ “Mudslide: A Spatially Anchored Census of Student Confusion for Online Lecture Videos."\\ \emph{ACM Conference on Human Factors in Computing Systems}.\\ %\textbf{Best of CHI Honorable Mention}.\\
\years{2015 CHI}J Kim, \textbf{EL Glassman}, A Monroy-Hernández, MR Morris.\\ “RIMES: Embedding Interactive Multimedia Exercises in Lecture Videos."\\ \emph{ACM Conference on Human Factors in Computing Systems}.\\
\years{2013 ICER}\textbf{EL Glassman}, N Gulley, RC Miller.\\ “Toward Facilitating Assistance to Students Attempting Engineering Design Problems."\\ \emph{ACM International Computing Education Research}.\\[.2cm]
%\emph{Underactuated robotics}\\
%\years{2012}\textbf{EL Glassman}, AL Desbiens, M Tobenkin, M Cutkosky, R Tedrake. ``Region of attraction estimation for a perching aircraft: A Lyapunov method exploiting barrier certificates.'' \emph{IEEE International Conference on Robotics and Automation (ICRA)}.\\
%\years{2010}\textbf{EL Glassman}, R Tedrake. ``A quadratic regulator-based heuristic for rapidly exploring state space.'' \emph{IEEE International Conference on Robotics and Automation (ICRA)}.\\[.2cm]
%\emph{Biomedical signal processing}\\
%\years{2006}\textbf{EL Glassman}, JV Guttag. ``Reducing the number of channels for an ambulatory patient-specific EEG-based epileptic seizure detector by applying recursive feature elimination.'' \emph{IEEE Engineering in Medicine and Biology Society (EMBS)}.\\

\subsection*{Technology Reports}
\noindent
%\emph{Interpretable Machine Learning}\\
\years{2015 MIT}B Kim, \textbf{EL Glassman}, B Johnson, J Shah.\\ ``iBCM: Interactive Bayesian Case Model Empowering Humans via Intuitive Interaction.''\\ MIT CSAIL TR-2015-010.

\subsection*{Book Chapters}
\noindent
%\emph{Learning at scale}\\
\years{2016 US Army}JJ Williams, J Kim, \textbf{EL Glassman}, A Rafferty, W Lasecki.\\ ``Making Static Lessons Adaptive through Crowdsourcing \& Machine Learning.''\\ \emph{Volume 4 of Design Recommendations for Intelligent Tutoring Systems}.\\ US Army Research Laboratory.

\subsection*{Theses}
\noindent
%\emph{Learning at scale}\\
\years{2016 MIT}\textbf{EL Glassman}.\\ ``Clustering and Visualizing Solution Variation in Massive Programming Classes.''\\ MIT EECS Ph.D. Thesis.\\[.2cm]
%\emph{Underactuated robotics}\\
%\years{2010}\textbf{EL Glassman}. ``A quadratic regulator-based heuristic for rapidly exploring state space.'' MIT EECS M.Eng. Thesis.\\

\subsection*{Poster and demo presentations}
\noindent
\years{2016 MSR}{\bf EL Glassman}. ``Learning Latent Student Design Decisions in Massive Python Programming Classes.'' \emph{New England Machine Learning Day}.\\
\years{2016 CSCW}{\bf EL Glassman}, RC Miller. ``Leveraging Learners for Teaching Programming and Hardware Design at Scale.'' \emph{ACM Computer-Supported Cooperative Work and Social Computing}.\\
\years{2015 L$@$S}{\bf EL Glassman}, CJ Terman, RC Miller. ``Learner-Sourcing in an Engineering Class at Scale.'' \emph{ACM Learning at Scale Conference}.\\
\years{2014 UIST}{\bf EL Glassman}. ``Interacting with massive numbers of student solutions.'' \emph{ACM Symposium on User Interface Software \& Technology}.\\
\years{2014 L$@$S}{\bf EL Glassman}, R Singh, RC Miller. ``Feature engineering for clustering student solutions.'' \emph{ACM Learning at Scale Conference}.\\

\section*{Prior Publications}

\subsection*{Underactuated robotics}
\noindent
\emph{Conference publications}\\
\years{2012 ICRA}\textbf{EL Glassman}, AL Desbiens, M Tobenkin, M Cutkosky, R Tedrake.\\ ``Region of attraction estimation for a perching aircraft: A Lyapunov method exploiting barrier certificates.''\\ \emph{IEEE International Conference on Robotics and Automation}.\\
\years{2010 ICRA}\textbf{EL Glassman}, R Tedrake.\\ ``A quadratic regulator-based heuristic for rapidly exploring state space.'' \\\emph{IEEE International Conference on Robotics and Automation}.\\[.2cm]
\emph{Posters}\\
\years{2009 NIPS}{\bf EL Glassman}. Women in Machine Learning Workshop, \emph{Neural Information Processing Systems}.\\
\emph{Theses}\\
\years{2010 MIT}\textbf{EL Glassman}.\\ ``A quadratic regulator-based heuristic for rapidly exploring state space.''\\ MIT EECS M.Eng. Thesis.\\


\subsection*{Biomedical signal processing}
\noindent
\emph{Journal articles}\\
\years{2005 TBME}\textbf{EL Glassman}.\\ “A wavelet-like filter based on neuron action potentials for analysis of human scalp electroencephalographs."\\ \emph{IEEE Transactions on Biomedical Engineering} 52 (11), 1851-1862.\\[.2cm]
\emph{Conference publications}\\
\years{2006 EMBS}\textbf{EL Glassman}, JV Guttag. \\``Reducing the number of channels for an ambulatory patient-specific EEG-based epileptic seizure detector by applying recursive feature elimination.''\\ \emph{IEEE Engineering in Medicine and Biology Society}.


\section*{Talks}

\subsection*{Seminars}
\noindent
%\years{2016}Computer Science Deptartment, Brown University \textit{(upcoming)}\\
\years{2016}Special Seminar for CS61a Staff, UC Berkeley's largest CS class\\
\years{2016}Berkeley Institute of Design, UC Berkeley\\
\years{2016}Thesis Defense, MIT CSAIL\\
\years{2015}Cooperation Group, Harvard Berkman Center\\
\years{2015}Computer Science Department, Duke University\\
\years{2015}Human-Computer Interaction summer lunch talk, Stanford University\\
\years{2015}HarvardX, Harvard University\\
\years{2015}Computer Science Department, Wellesley College\\
\years{2014}DUB Seminar, HCI \& Design, University of Washington\\
\years{2001}Special Seminar, Schlumberger-Doll Research Center\\

\subsection*{Conference presentations}
\noindent
\years{2016}DocMatrix: Self-Teaching from Multiple Sources. \\\emph{ASIS\&T Annual
Meeting}, Copenhagen.\\
\years{2016} Learnersourcing Personalized Hints. \\\emph{ACM CSCW}, San Francisco.\\
\years{2015} Foobaz: Variable Name Feedback for Student Code at Scale. \\\emph{ACM UIST}, Charlotte NC.\\
\years{2015}Mudslide: A Spatially Anchored Census of Student Confusion for Online Lecture Videos. \\\emph{ACM CHI}, Seoul.\\
\years{2015}OverCode: Visualizing variation in student solutions to programming problems at scale. \\\emph{ACM CHI}, Seoul.\\
\years{2013}Toward Facilitating Assistance to Students Attempting Engineering Design Problems. \\\emph{ACM ICER}, San Diego.\\
\years{2012}Region of attraction estimation for a perching aircraft: A Lyapunov method exploiting barrier certificates. \\\emph{IEEE ICRA}, St. Paul.\\
\years{2010}A quadratic regulator-based heuristic for rapidly exploring state space. \\\emph{IEEE ICRA}, Anchorage.\\
\years{2006}Reducing the number of channels for an ambulatory patient-specific EEG-based epileptic seizure detector by applying recursive feature elimination. \\\emph{IEEE EMBS}, New York City.

\subsection*{Workshops}
\noindent
\years{2016} ``Learning Latent Student Design Decisions in Python Programming Classes.'' Workshop on Machine Learning for Digital Education and Assessment Systems, \emph{International Conference on Machine Learning (ICML)}.\\
\years{2016} Tools for Thought, Recurse Center, NYC.\\
\years{2015} Rising Stars Workshop for aspiring CS faculty, MIT.\\
\years{2015} ``Interacting with massive numbers of student solutions.'' Doctoral consortium, \emph{ACM Symposium on User Interface Software \& Technology (UIST)}.\\
\years{2013} ``Visualizing and classifying multiple solutions to engineering design problems.'' Doctoral consortium, \emph{ACM International Computing Education Research (ICER)}.




\section*{Selected Press}
%\noindent
\years{2015}\emph{MIT News Homepage Spotlight}, ``Reviewing online homework at scale'' (research profile)\\
\years{2015}\emph{Reddit's Upvoted podcast} guest\\
\years{2014}\emph{WIRED} opinion piece, ``MIT Computer Scientists Demonstrate the Hard Way That Gender Still Matters'' co-author\\
\years{2004}\emph{New York Times}, ``Not Too Young for a Patent'' (personal profile)\\
\years{2003}\emph{CNN} Lou Dobbs Tonight, ``America's Bright Future'' (personal profile)\\
\years{2003}\emph{CNN} American Morning guest\\
\years{2003}\emph{Science} ``Rising Stars'' Vol. 300. Issue 5624, pp. 1368 (personal profile)\\



%\hrule
%\section*{Certifications}
%\years{2011}Graduate Student Teaching Certificate, MIT Teaching \& Learning Lab\\
%\years{2002}Amatuer (Ham) Radio License, American Radio Relay League (ARRL)\\

%\hrule
\section*{Leadership}

\subsection*{Workshops}
\noindent
\years{2017}Co-Organizer. ``Program Synthesis Hackathon'' with the Microsoft Program Synthesis using Examples SDK (PROSE), UC Berkeley.

\subsection*{Research mentoring}
\noindent
\years{2016}Hezheng Yin \hfill UC Berkeley EECS Ph.D. student\\
\years{2016}Andrew Head \hfill UC Berkeley EECS Ph.D. student\\
\years{2016}Eric Pai \hfill UC Berkeley EECS undergraduate\\
\years{2016}Michelle Tian \hfill UC Berkeley EECS undergraduate\\
\years{2016}Daniel Nguyen \hfill UC Berkeley EECS undergraduate\\
\years{2016}Sindy Tan \hfill Harvard EECS undergraduate\\
\years{2015-2016}Stacey Terman \hfill MIT EECS M.Eng. student\\
%\years{2012}Co-organizer, MIT edTech reading group\\
\years{2015}Aaron Lin \hfill MIT EECS undergraduate\\

\subsection*{Outreach}
\noindent
\years{2016}Panelist, MIT EECS SuperUROP (Undergraduate Research) Seminar\\
\years{2015}Invited speaker, GirlTechPower summer camp for girls\\
\years{2015}Panelist, Women Techmaker's Summit at Google Cambridge\\
\years{2014-2015}Invited speaker, MIT CSAIL Hour of Code event for local schools\\
\years{2014}Reddit AMA on gender, CS, and academia with Jean Yang and Neha Nerula\\
\years{2013}Mentor, Harvard Women in CS ``Women Engineers Code Hackathon''\\
\years{2013}Panelist, MIT EECS Teaching Assistant Orientation\\
\years{2011}MIT Robot Locomotion Group representative, Cambridge Science Festival\\
\years{2011}MIT Robot Locomotion Group representative, New Hampshire TechFest\\
\years{2008, 2011}Invited speaker, MIT Women's Technology Program\\
\years{2008}Invited speaker, MIT CSAIL Campus Preview Weekend\\

\subsection*{MIT student groups}
\noindent
\years{2013-2015}President \hfill Middle East Education through Technology\\
%\years{2012}Co-organizer, MIT edTech reading group\\
\years{2008-2009}Vice-President \hfill Eta Kappa Nu EECS honor society\\

%\hrule


\section*{Service}
%\hrule
\subsection*{Department}
\noindent
\years{2006-2008}MIT EECS Department Education Committee member\\
\years{2005}MIT Council on Educational Technology member\\

%\hrule
\subsection*{Profession}
\noindent
\years{2017}ACM UIST Registration Chair\\
\years{2015-present}ACM CHI, UIST, CSCW reviewer\\
%\years{2016}ACM CSCW reviewer\\
%\years{2015}ACM UIST reviewer\\
\years{2015}ACM CHI session chair, social media \& citizen science\\
\years{2015}ACM CHI Works-in-Progress Program Committee member\\






%\vspace{1cm}
\vfill{}
%\hrulefill

\begin{center}
{\scriptsize  Last updated: \today\- •\- 
% ---- PLEASE LEAVE THIS BACKLINK FOR ATTRIBUTION AS PER CC-LICENSE
Typeset in \href{http://nitens.org/taraborelli/cvtex}{
%\fontspec{Times New Roman}
\XeTeX }\\
% ---- FILL IN THE FULL URL TO YOUR CV HERE
\href{http://eglassman.github.io/CV.pdf}{http://eglassman.github.io/CV.pdf}}
\end{center}

\end{document}