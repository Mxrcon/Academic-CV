%------------------------------------
% Dario Taraborelli
% Typesetting your academic CV in LaTeX
%
% URL: http://nitens.org/taraborelli/cvtex
% DISCLAIMER: This template is provided for free and without any guarantee 
% that it will correctly compile on your system if you have a non-standard  
% configuration.
% Some rights reserved: http://creativecommons.org/licenses/by-sa/3.0/
%------------------------------------

%!TEX TS-program = xelatex
%!TEX encoding = UTF-8 Unicode

\documentclass[10pt, a4paper]{article}
\usepackage{fontspec} 

% DOCUMENT LAYOUT
\usepackage{geometry} 
\geometry{a4paper, textwidth=5.5in, textheight=8.5in, marginparsep=7pt, marginparwidth=.75in}
\setlength\parindent{0in}

% FONTS
\usepackage[usenames,dvipsnames]{xcolor}
\usepackage{xunicode}
\usepackage{xltxtra}
\usepackage{multicol}
\setlength{\columnsep}{1cm}
\defaultfontfeatures{Mapping=tex-text}
%\setromanfont [Ligatures={Common}, Numbers={OldStyle}, Variant=01]{Linux Libertine O}
%\setmonofont[Scale=0.8]{Monaco}
%%% modified by Karol Kozioł for ShareLaTeX use
\setmainfont[
  Ligatures={Common}, Numbers={OldStyle}, Variant=01,
  BoldFont=LinLibertine_RB.otf,
  ItalicFont=LinLibertine_RI.otf,
  BoldItalicFont=LinLibertine_RBI.otf
]{LinLibertine_R.otf}
\setmonofont[Scale=0.8]{DejaVuSansMono.ttf}

% ---- CUSTOM COMMANDS
\chardef\&="E050
\newcommand{\html}[1]{\href{#1}{\scriptsize\textsc{[html]}}}
\newcommand{\pdf}[1]{\href{#1}{\scriptsize\textsc{[pdf]}}}
\newcommand{\doi}[1]{\href{#1}{\scriptsize\textsc{[doi]}}}
% ---- MARGIN YEARS
\usepackage{marginnote}
\newcommand{\amper{}}{\chardef\amper="E0BD }
\newcommand{\years}[1]{\marginnote{\normalsize #1}}
\renewcommand*{\raggedleftmarginnote}{}
\setlength{\marginparsep}{7pt}
\reversemarginpar

% HEADINGS
\usepackage{sectsty} 
\usepackage[normalem]{ulem} 
\sectionfont{\mdseries\upshape\Large}
\subsectionfont{\mdseries\scshape\normalsize} 
\subsubsectionfont{\mdseries\upshape\large} 

% PDF SETUP
% ---- FILL IN HERE THE DOC TITLE AND AUTHOR
\usepackage[%dvipdfm, 
bookmarks, colorlinks, breaklinks, 
% ---- FILL IN HERE THE TITLE AND AUTHOR
	pdftitle={Glassman CV},
	pdfauthor={Elena Leah Glassman},
	pdfproducer={http://nitens.org/taraborelli/cvtex}
]{hyperref}  
\hypersetup{linkcolor=blue,citecolor=blue,filecolor=black,urlcolor=MidnightBlue} 

% DOCUMENT
\begin{document}
{\LARGE Elena Leah Glassman}\\[0.5cm]
University of California, Berkeley\\
%354 Hearst Memorial Mining Building\\
Berkeley, CA USA\\%[.2cm] \texttt{94720} 
\texttt{+1 215-694-9631}\\
 %\\%[.2cm]
\href{mailto:eglassman@berkeley.edu }{eglassman@berkeley.edu } \\
\href{http://eglassman.github.io/}{eglassman.github.io}\\ 

%\vfill
%Born:  March 12, 1879---Ulm, Germany\\
%Nationality:  German/American

%%\hrule
% \section*{Current position}
% \emph{Postdoctoral Scholar}, Berkeley Institute of Design, UC Berkeley EECS

%%\hrule
\section*{Areas of specialization}
Human-Computer Interaction • Programming Systems/Software • Data Science \\• Human \& machine teaching  

%%\hrule
\section*{Academic Research Positions}
\noindent

\years{2016-present}EECS Postdoctoral Scholar, Berkeley Institute of Design \hfill UC Berkeley\\
\emph{Supervisor: Bj{\"o}rn Hartmann, Associate Professor of EECS}\\
\emph{Funded by NSF Expeditions in Computer Augmented Program Engineering (ExCAPE) grant}\\
\years{2012-2016}Graduate researcher, User Interface Design Group \hfill CS \& AI Lab, MIT\\
\emph{Advisor: Robert Miller, Professor of Computer Science}\\
\years{2010-2011}Visiting researcher, Biomimetics \& Dexterous Manipulation Lab\hfill Stanford University\\
\years{2008-2011}Graduate researcher, Robot Locomotion Group\hfill CS \& AI Lab, MIT\\
\years{2004-2008}Undergraduate researcher\hfill CS \& AI Lab, MIT\\
\years{2003-2004}Student researcher (invited), EEG Lab\hfill Princeton University

% \years{2016-present}\textbf{University of California, Berkeley}, Postdoctoral Scholar \hfill Berkeley Institute of Design\\ 
% Supervisor: Bj{\"o}rn Hartmann, Associate Professor of EECS\\
% \years{2012-2016}\textbf{MIT}, EECS PhD Candidate \hfill CSAIL User Interface Design Group\\ %\hfill CS \& AI Lab, MIT\\
% Advisor: Robert C. Miller, Professor of Computer Science\\
% \years{2010-2011}\textbf{Stanford University}, Visiting researcher \hfill Biomimetics \& Dexterous Manipulation Lab\\
% Host: Mark R. Cutkosky, Professor of Mechanical Engineering\\
% \years{2008-2011}\textbf{MIT}, EECS Graduate researcher \hfill CSAIL Robot Locomotion Group\\
% Advisor: Russ Tedrake, Professor of EECS\\
% \years{2004-2008}\textbf{MIT}, EECS Undergraduate researcher\hfill CSAIL\\
% Advisors: John Guttag and Russ Tedrake, Professors of EECS\\
% \years{2003-2004}\textbf{Princeton University}, Volunteer researcher \hfill Psychology Dept. EEG Lab\\
% Host: Jack Gelfand, Research Scientist

%%\hrule
\section*{Industry Research Positions}
\noindent
%\years{2016}Research scientist (contractor) \hfill Search, Google\\
\years{2015}User experience research intern \hfill Search, Google\\
\emph{Advisor: Dan Russell, Senior Research Scientist}\\
\years{2014}Design research intern \hfill neXus Research Team, Microsoft Research\\
\emph{Advisors: Meredith Ringel Morris, Principal Researcher, and Andrés Monroy-Hernández, Researcher}\\

%\hrule
\section*{Education}
\noindent
\years{2016 MIT}\textsc{Ph.D.}, Electrical Engineering \& Computer Science\hfill Cambridge, MA\\
%``Clustering and Visualizing Solution Variation in Massive Programming Classes''\\
%\emph{Advisor: Robert C. Miller, Professor of Computer Science}\\
\years{2010 MIT}\textsc{M.Eng.}, Electrical Engineering \& Computer Science\hfill Cambridge, MA\\
%``A quadratic regulator-based heuristic for rapidly exploring state space''\\
%\emph{Advisor: Russ Tedrake, Professor of Electrical Engineering \& Computer Science}\\
\years{2008 MIT}\textsc{B.S.}, Electrical Science \& Engineering\hfill Cambridge, MA\\

\section*{Selected fellowships and scholarships}
\noindent
\years{2017}Moore/Sloan Data Science Fellowship at the Berkeley Institute for Data Science (BIDS)\\
\years{2014}MIT Amar Bose Teaching Fellowship, for developing innovative tools for teaching CS at scale\\
\years{2011-2014}NSF Graduate Research Fellow (NSF GRFP)\\
\years{2008-2011}National Defense Science and Engineering Graduate Fellow (NDSEG)\\
\years{2004} IEEE President's Scholarship (\$10,000)\\
\years{2003} Intel Foundation Young Scientist Award (\$50,000) \\ \emph{Awarded to the top 3 individual projects at the Intel International Science \& Engineering Fair}


%\hrule
\section*{Selected honors \& awards}
\noindent
%\years{2015}Invited research presentation, Rising Stars workshop for aspiring CS faculty, MIT
\years{2016}Audience Choice Award, MIT Can Talk speech competition\\
\years{2015}Best of CHI Honorable Mention (top 5\% of papers) \\
\years{2015}Research talk at MIT's Rising Stars workshop for aspiring CS faculty\\
\years{2009}Masterworks Oral Thesis Presentation Award, MIT EECS\\
\years{2008}Vice President and member, Eta Kappa Nu, EECS Honor Society\\
\years{2004}Valedictorian \& commencement speaker, Central Bucks High School West\\
\years{2004}National Gallery for America's Young Inventors\\
\years{2003} Intel International Science and Engineering Fair -- Best of Category: Computer Science (\$5,000)\\


\section*{Selected press}
%\noindent
\years{2015 MIT}\emph{MIT News Homepage Spotlight}, ``Reviewing online homework at scale'' (research profile).\\
\years{2015 Reddit}\emph{Reddit's Upvoted podcast} guest.\\
\years{2014 WIRED}\emph{WIRED} opinion piece, ``MIT Computer Scientists Demonstrate the Hard Way That Gender Still Matters'' co-author.\\
\years{2004 NYT}\emph{New York Times}, ``Not Too Young for a Patent'' (personal profile).\\
\years{2003 CNN}\emph{CNN} Lou Dobbs Tonight, ``America's Bright Future'' (personal profile).\\
\years{2003 CNN}\emph{CNN} American Morning guest.\\
\years{2003 Science}\emph{Science} ``Rising Stars'' Vol. 300. Issue 5624, p. 1368 (personal profile).\\




\section*{Publications}


\subsection*{Theses}
\noindent
\years{2016 MIT}``Clustering and Visualizing Solution Variation in Massive Programming Classes''\\ %\textbf{E Glassman}\\ 
\textsc{Ph.D.} Thesis, MIT Electrical Engineering \& Computer Science. \\\\
\years{2010 MIT}``A Quadratic Regulator-based Heuristic for Rapidly Exploring State Space''\\
%\textbf{E Glassman} \\
\textsc{M.Eng.} Thesis, MIT Electrical Engineering \& Computer Science.\\

\subsection*{Journal articles}
\noindent
%\emph{Learning at scale}\\
\years{2015 TOCHI}
\textbf{E Glassman}, J Scott, R Singh, P Guo, RC Miller\\
``OverCode: visualizing variation in student solutions to programming problems at scale''\\ 
\emph{ACM Transactions on Computer-Human Interaction}, 22 (2), April 2015.\\
{\small Special Issue on Online Learning at Scale}\\\\
\years{2005 TBME}
\textbf{E Glassman}\\ 
``A wavelet-like filter based on neuron action potentials for analysis of human scalp electroencephalographs''\\
\emph{IEEE Transactions on Biomedical Engineering} 52 (11), 1851-1862, Nov. 2005.\\
%Online Learning at Scale Special Issue
%\emph{Biomedical signal processing}\\
%\years{2005}\textbf{EL Glassman}. “A wavelet-like filter based on neuron action potentials for analysis of human scalp electroencephalographs." \emph{IEEE Transactions on Biomedical Engineering (TBME)} 52 (11), 1851-1862.\\

\subsection*{Refereed conference papers}

\small{Top-tier ACM conferences in human-computer interaction, i.e., CHI, CSCW, and UIST, are highly selective venues intended for archival papers only. These conferences are comparable to or exceed many IEEE journals in their selectivity, visibility, and impact.} 
%For more details, see ``Selectivity and Impact'' by Jilin Chen and Joseph Konstan.}
\\\\
\noindent
%\emph{Learning at scale}\\
\years{2018 CHI}\textbf{E Glassman}*, T Zhang*, B Hartmann, and M Kim\\
``Visualizing API Usage Examples at Scale''\\
\emph{ACM Conference on Human Factors in Computing Systems} (CHI), 2018.\\
\emph{20-25\% acceptance rate (exact figure to be released)}\\\\
\years{2018 CHI} A Head, \textbf{E Glassman}, B Hartmann, and M Hearst\\
``Interactive Extraction of Examples from Existing Code''\\
\emph{ACM Conference on Human Factors in Computing Systems} (CHI), 2018.\\ 
\emph{20-25\% acceptance rate (exact figure to be released)}\\\\
%\years{2017}Writing Reusable Code Feedback at Scale with Mixed-Initiative Program Synthesis\\A Head, \textbf{E Glassman}, G Soares, R Suzuki, L Figueredo, L D'Antoni and B Hartmann\\  ACM Learning at Scale\\
%\emph{13\% acceptance rate}\\\\
\years{2017 L@S}A Head, \textbf{E Glassman}, G Soares, R Suzuki, L Figueredo, L D'Antoni and B Hartmann\\
``Writing Reusable Code Feedback at Scale with Mixed-Initiative Program Synthesis''\\
\emph{ACM Learning at Scale} (L@S), 2017.\\
\emph{13\% acceptance rate}\\\\
\years{2017 VL/HCC}R Suzuki, G Soares, A Head, \textbf{E Glassman}, R Reis, M Mongiovi, L D'Antoni, and B Hartmann\\
``TraceDiff: Debugging Unexpected Code Behavior Using Trace Divergences''\\ 
\emph{IEEE Symposium on Visual Languages and Human-Centric Computing} (VL/HCC), 2017.\\
\emph{29\% acceptance rate}\\\\
\years{2016 CSCW}
\textbf{E Glassman}, A Lin, C Cai, R Miller\\
``Learnersourcing Personalized Hints''\\  
\emph{ACM Computer-Supported Cooperative Work and Social Computing} (CSCW), 2017.\\
\emph{25\% acceptance rate}\\\\
\years{2016 ASIST}
\textbf{E Glassman}, D Russell\\ 
``DocMatrix: Self-Teaching from Multiple Sources''\\ 
ASIS\&T Annual Meeting, 2016.\\
\emph{40\% acceptance rate}\\\\
\years{2015 UIST}\textbf{E Glassman}, L Fischer, J Scott, R Miller\\ 
``Foobaz: Variable Name Feedback for Student Code at Scale''\\
\emph{ACM Symposium on User Interface Software \& Technology} (UIST), 2015.\\
\emph{23.6\% acceptance rate}\\\\
%\newpage
\years{2015 CHI}\textbf{Best of CHI Honorable Mention (top 5\%)}\\
\textbf{E Glassman}, J Kim, A Monroy-Hernández, MR Morris\\
``Mudslide: A Spatially Anchored Census of Student Confusion for Online Lecture Videos''\\   
\emph{ACM Conference on Human Factors in Computing Systems} (CHI), 2015.\\ 
\emph{23\% acceptance rate}\\\\
\years{2015 CHI}
J Kim, \textbf{EL Glassman}, A Monroy-Hernández, MR Morris\\ 
``RIMES: Embedding Interactive Multimedia Exercises in Lecture Videos''\\  
\emph{ACM Conference on Human Factors in Computing Systems} (CHI), 2015.\\ 
\emph{23\% acceptance rate}\\\\
\years{2013 ICER}
\textbf{E Glassman}, N Gulley, RC Miller\\
``Toward Facilitating Assistance to Students Attempting Engineering Design Problems''\\  
\emph{ACM International Computing Education Research} (ICER), 2013.\\
\emph{33\% acceptance rate}\\\\
\years{2012 ICRA}
\textbf{E Glassman}, A Desbiens, M Tobenkin, M Cutkosky, and R Tedrake\\
``Region of attraction estimation for a perching aircraft: A Lyapunov method exploiting barrier certificates''\\
\emph{IEEE International Conference on Robotics and Automation} (ICRA), 2012.\\
\emph{40\% acceptance rate}\\\\
\years{2010 ICRA}
\textbf{E Glassman} and R Tedrake\\
``A quadratic regulator-based heuristic for rapidly exploring state space''\\
\emph{IEEE International Conference on Robotics and Automation} (ICRA), 2010.\\\\
\years{2006 EMBS}
\textbf{E Glassman} and J Guttag \\
``Reducing the number of channels for an ambulatory patient-specific EEG-based epileptic seizure detector by applying recursive feature elimination''\\
\emph{IEEE Engineering in Medicine and Biology Society} (EMBS), 2006.

%[.2cm]
%\emph{Underactuated robotics}\\
%\years{2012}\textbf{EL Glassman}, AL Desbiens, M Tobenkin, M Cutkosky, R Tedrake. ``Region of attraction estimation for a perching aircraft: A Lyapunov method exploiting barrier certificates.'' \emph{IEEE International Conference on Robotics and Automation (ICRA)}.\\
%\years{2010}\textbf{EL Glassman}, R Tedrake. ``A quadratic regulator-based heuristic for rapidly exploring state space.'' \emph{IEEE International Conference on Robotics and Automation (ICRA)}.\\[.2cm]
%\emph{Biomedical signal processing}\\
%\years{2006}\textbf{EL Glassman}, JV Guttag. ``Reducing the number of channels for an ambulatory patient-specific EEG-based epileptic seizure detector by applying recursive feature elimination.'' \emph{IEEE Engineering in Medicine and Biology Society (EMBS)}.\\

\subsection*{MIT Technology Reports}
\noindent
%\emph{Interpretable Machine Learning}\\
\years{2015 CSAIL}
B Kim, \textbf{E Glassman}, B Johnson, J Shah\\
``iBCM: Interactive Bayesian Case Model Empowering Humans via Intuitive Interaction''\\  MIT CSAIL TR-2015-010, April 2015.

\subsection*{Book Chapters}
\noindent
%\emph{Learning at scale}\\
\years{2016}
JJ Williams, J Kim, \textbf{E Glassman}, A Rafferty, W Lasecki\\
``Making Static Lessons Adaptive through Crowdsourcing \& Machine Learning''\\ 
\emph{Design Recommendations for Intelligent Tutoring Systems: Domain Modeling} Vol. 4, \\
US Army Research Laboratory, July 2016.
%\\US Army Research Laboratory

%\subsection*{Theses}
%\noindent
%\emph{Learning at scale}\\
%\years{2016 MIT}Clustering and Visualizing Solution Variation in Massive Programming Classes\\ \textbf{E Glassman}\\ MIT EECS Ph.D. Thesis\\[.2cm]
%\emph{Underactuated robotics}\\
%\years{2010}\textbf{EL Glassman}. ``A quadratic regulator-based heuristic for rapidly exploring state space.'' MIT EECS M.Eng. Thesis.\\

\subsection*{Posters and demos}
\noindent
\years{2017 CHI}
R Suzuki, G Soares, {\bf E Glassman}, A Head, L D'Antoni, and B Hartmann\\
``Exploring the Design Space of Automatically Synthesized Hints for Introductory Programming Assignments'' \\ \emph{ACM CHI Conference on Human Factors in Computing Systems} (CHI), 2017.\\\\
\years{2017 L@S} 
A Ju, {\bf E Glassman}, A Fox\\
``Teamscope: Scalable Team Evaluation via Automated Metric Mining for Communication, Organization, Execution, and Evolution''\\
\emph{ACM Learning at Scale Conference} (L@S), 2017.\\\\
\years{2016 ICML}
{\bf E Glassman} \\
``Learning Latent Student Design Decisions in Python Programming Classes''\\
Workshop on Machine Learning for Digital Education and Assessment Systems\\
\emph{International Conference on Machine Learning} (ICML), 2016.\\\\
\years{2016 NEML}
{\bf E Glassman} \\
``Learning Latent Student Design Decisions in Massive Python Programming Classes''\\
\emph{New England Machine Learning Day} (NEML), 2016.\\\\
\years{2016 CSCW}
{\bf E Glassman} and R Miller \\
``Leveraging Learners for Teaching Programming and Hardware Design at Scale''\\
\emph{ACM Computer-Supported Cooperative Work and Social Computing} (CSCW), 2016.\\\\
\years{2016 CSCW}
{\bf E Glassman}, B Kim, J Shah\\
``Scaling Up Qualitative Data Analysis With Interfaces Powered by Interpretable Machine Learning'' \\Human Centered Data Science Workshop\\
\emph{ACM Computer-Supported Cooperative Work and Social Computing} (CSCW), 2016.\\\\
%\years{2016 MIT} MIT CSAIL Research Party, Cambridge, MA.\\
%\years{2015 MIT}{\bf EL Glassman}. Rising Stars Workshop for aspiring CS faculty, MIT.\\
%\newpage
\years{2015 L$@$S}
{\bf E Glassman}, C Terman, R Miller\\
``Learner-Sourcing in an Engineering Class at Scale''\\
\emph{ACM Learning at Scale Conference} (L@S), 2015.\\\\
\years{2014 UIST}
{\bf E Glassman}\\
``Interacting with Massive Numbers of Student Solutions''\\
\emph{ACM Symposium on User Interface Software \& Technology} (UIST), 2014.\\\\
\years{2014 L$@$S}
{\bf E Glassman}, R Singh, R Miller\\
``Feature Engineering for Clustering Student Solutions''\\
\emph{ACM Learning at Scale Conference} (L@S), 2014.\\\\
\years{2009 NIPS}
{\bf E Glassman}\\
``A quadratic regulator-based heuristic for rapidly exploring state space''\\
Women in Machine Learning Workshop (WIML)\\
\emph{Neural Information Processing Systems} (NIPS), 2009\\


\section*{Service}
%\hrule


%\hrule
\subsection*{Program committees}
\noindent
\years{2017}ACM CHI, Engineering Interactive Systems and Technologies subcommittee\\
\years{2017}ACM Learning at Scale (L@S)\\
\years{2017}SPLASH Workshop on Evaluation and Usability of Programming Languages and Tools (PLATEAU)\\
\years{2015}ACM CHI Works-in-Progress

\subsection*{Organizing chairs}
\noindent
\years{2017}ACM UIST Registration Chair\\
%\years{2016}ACM CSCW reviewer\\
%\years{2015}ACM UIST reviewer\\
\years{2017}ACM UIST session chair, ``Code/Education Session''\\
\years{2015, 2017}ACM CHI session chair, ``Social Media \& Citizen Science'' and ``All About Data''

\subsection*{Reviewing}
\noindent
\years{2017}ACM Transactions on Computer-Human Interaction (TOCHI)\\
%\years{2017}ACM CHI session chair, ``All About Data''\\
\years{2015-present}ACM CHI, UIST, and CSCW

\subsection*{Department and institute committees}
\noindent
\years{2006-2008}MIT EECS Department Education Committee member\\
\years{2005}MIT Council on Educational Technology member




\section*{Teaching}

\subsection*{Experience}

\years{2016}Co-lecturer, User Interface Design \& Implementation ($\approx$ 175 students) \hfill MIT EECS\\
\years{2013}Instructor, introductory Python programming \hfill MIT MEET, Jerusalem\\
\years{2013}Educational video script writer \& presenter, radio receiver technology \hfill MIT Teaching \& Learning Lab\\
\years{2012-2014}Teaching assistant, Computation Structures\hfill MIT EECS\\
\years{2011}Teaching assistant, Introduction to EECS 1\hfill MIT EECS\\
\years{2006-2011}Tutor, Signals, Systems, \& Probabilistic Systems Analysis\hfill MIT EECS Honor Society

\subsection*{Certifications}

\years{2011}Graduate Student Teaching Certificate\hfill MIT Teaching \& Learning Lab

\section*{Invited Talks}

%\subsection*{Seminar Presentations}
\noindent
%\years{2017 UPenn}NSF ExCAPE PI Meeting\\
\years{2017}ACM KDD Workshop on Advancing Education with Data \hfill Halifax, Nova Scotia \\
\years{2017} Stanford HCI summer seminar \hfill Stanford, CA\\
\years{2017} MIT CSAIL Machine Learning Tea\hfill Cambridge, MA\\
\years{2016} Special Seminar for CS61a Staff, UC Berkeley's largest CS class\hfill Berkeley, CA\\
\years{2016}Berkeley Institute of Design \hfill Berkeley, CA\\
%\years{2016 MIT}Thesis Defense, CSAIL\\
\years{2015}Harvard Berkman Center Cooperation Group\hfill Cambridge, MA\\
\years{2015}Duke Computer Science Department seminar \hfill Durham, NC\\%, Duke University\\
\years{2015}Stanford HCI summer seminar\hfill Stanford, CA\\
\years{2015}HarvardX\hfill Cambridge, MA\\
\years{2015}Wellesley Computer Science Department  seminar\hfill Wellesley, MA\\
\years{2014}DUB Seminar on HCI \& Design, University of Washington\hfill Seattle, WA\\
\years{2001}Special Seminar, Schlumberger-Doll Research Center\hfill Ridgefield, CT

\section*{Invitation-only workshops, seminars, and conferences}

\subsection*{DARPA}
\noindent
\years{2017}Speaker, Diverse Ways of Inferring Missions \hfill Washington, D.C.\\
\years{2017}Augmented Developers: Tools for Hybrid Human-Machine Software Eng. \hfill Washington, D.C.

\subsection*{Schloss Dagstuhl – Leibniz Center for Informatics}
\noindent
\years{2017}Speaker, Approaches and Applications of Inductive Programming \hfill Wadern, Germany

\subsection*{NSF-funded groups}
\noindent
\years{2017}Speaker, Expeditions in Computer Augmented Program Engineering (ExCAPE)\\ NSF grant PI meeting \hfill Philadelphia, PA\\
\years{2017}Community-building for data-intensive computer \& computing science\\ education infrastructure research (SPLICE) organizational meeting \hfill Pittsburgh, PA

\subsection*{Independent research organizations}
\noindent
\years{2017}Moore-Sloan Data Science Summit hosted by Moore-Sloan Foundation \hfill New Orleans, LA\\
\years{2017}Y Conf hosted by Y Combinator Research \hfill San Francisco, CA\\
\years{2016} Speaker, Tools for Thought, Recurse Center \hfill NYC, NY

\subsection*{Doctoral Consortiums}
\noindent
\years{2015}ACM UIST, ``Interacting with massive numbers of student solutions'' \hfill Honolulu, HI \\ 
\years{2013}ACM ICER, ``Visualizing \& classifying multiple solutions to engineering \\design problems''\hfill San Diego, CA



%\pagebreak
%\section*{Publications in Other Fields}

%\subsection*{Underactuated robotics}
%\noindent
%\emph{Conference publications}\\

%\emph{Posters}\\

%\emph{Theses}\\



%\subsection*{Biomedical signal processing}
%\noindent
%\emph{Journal articles}\\
%\years{2005 TBME}A wavelet-like filter based on neuron action potentials for analysis of human scalp electroencephalographs\\\textbf{E Glassman}\\ \emph{IEEE Transactions on Biomedical Engineering} 52 (11), 1851-1862.\\[.2cm]
%\emph{Conference publications}\\


%\newpage


% \subsection*{Conference presentations}
% \noindent
% \years{2016 ASIS\&T}DocMatrix: Self-Teaching from Multiple Sources. \emph{ASIS\&T Annual
% Meeting}, Copenhagen.\\
% \years{2016 CSCW} Learnersourcing Personalized Hints. \emph{ACM CSCW}, San Francisco.\\
% \years{2015 UIST} Foobaz: Variable Name Feedback for Student Code at Scale. \\\emph{ACM UIST}, Charlotte NC.\\
% \years{2015 CHI}Mudslide: A Spatially Anchored Census of Student Confusion for Online Lecture Videos. \emph{ACM CHI}, Seoul.\\
% \years{2015 CHI}OverCode: Visualizing variation in student solutions to programming problems at scale. \emph{ACM CHI}, Seoul.\\
% \years{2013 ICER}Toward Facilitating Assistance to Students Attempting Engineering Design Problems. \emph{ACM ICER}, San Diego.\\
% \years{2012 ICRA}Region of attraction estimation for a perching aircraft: A Lyapunov method exploiting barrier certificates. \emph{IEEE ICRA}, St. Paul.\\
% \years{2010 ICRA}A quadratic regulator-based heuristic for rapidly exploring state space. \emph{IEEE ICRA}, Anchorage.\\
% \years{2006 EMBS}Reducing the number of channels for an ambulatory patient-specific EEG-based epileptic seizure detector by applying recursive feature elimination. \emph{IEEE EMBS}, New York City.





%\hrule
%\section*{Certifications}
%\years{2011}Graduate Student Teaching Certificate, MIT Teaching \& Learning Lab\\
%\years{2002}Amatuer (Ham) Radio License, American Radio Relay League (ARRL)\\

%\hrule
\section*{Leadership}

\subsection*{Hackathons, Student Groups, and Reading Groups}
\noindent
\years{2017}Co-organizer, Text Across Domains (TextXD) Workshop \hfill Berkeley Institute of Data Science\\
\years{2017}Co-organizer, Program Synthesis Hackathon \hfill UC Berkeley\\
\years{2013-2015}President, Middle East Education through Technology \hfill MIT\\
\years{2012} Co-organizer, edTech reading group \hfill MIT

\subsection*{Research mentoring}
\noindent
\years{2017}Kunal Chaudhary, EECS undergraduate \hfill UC Berkeley\\
\years{2017}Julie Deng, EECS \& Cognitive Science undergraduate \hfill UC Berkeley\\
\years{2017}Orkun Duman, EECS undergraduate \hfill UC Berkeley \\
%\years{2017} Emily Pedersen, EECS undergraduate \hfill UC Berkeley\\
\years{2016-17}Hezheng Yin, EECS Ph.D. student \hfill UC Berkeley\\
%{\it Worked on yet-unpublished text information visualization project}\\
\years{2016-17}Andrew Head, EECS Ph.D. student, {\it co-author} \hfill UC Berkeley\\
%{\it Project supervisor for OverCode deployment and Master's thesis}\\
\years{2016-17}Eric Pai,  EECS undergraduate and Master's student \hfill UC Berkeley\\
{\it Project supervisor for OverCode deployment and Master's thesis}\\
%\years{2016}Michelle Tian \hfill UC Berkeley EECS undergraduate\\
%\years{2016}Daniel Nguyen \hfill UC Berkeley EECS undergraduate\\
\years{2016-17}Sindy Tan, EECS undergraduate \hfill Harvard\\
{\it Co-advised senior student research experience}\\
\years{2015-16}Stacey Terman, EECS M.Eng. student \hfill MIT\\
{\it Supervised Master's thesis}\\
%\years{2012}Co-organizer, MIT edTech reading group\\
\years{2015}Aaron Lin, EECS undergraduate, {\it co-author} \hfill MIT
%{\it Co-author}\\


%\subsection*{MIT student groups}
%\noindent
%\years{2013-2015}President \hfill Middle East Education through Technology\\
%\years{2012}Co-organizer, MIT edTech reading group\\
%\years{2008-2009}Vice-President \hfill Eta Kappa Nu EECS honor society

\subsection*{Selected Outreach}
\noindent
\years{2016}Panelist, MIT EECS SuperUROP (Undergraduate Research) Seminar\\
\years{2016}Virtual guest speaker, Bucknell HCI course\\
\years{2015}Invited speaker, GirlTechPower summer camp for girls\\
\years{2015}Panelist, Women Techmaker's Summit at Google Cambridge\\
\years{2014-2015}Invited speaker, MIT CSAIL Hour of Code event for local schools\\
\years{2014}Reddit AMA on gender, CS, and academia with Jean Yang and Neha Nerula\\
\years{2013}Mentor, Harvard Women in CS ``Women Engineers Code Hackathon''\\
\years{2013}Panelist, MIT EECS Teaching Assistant Orientation\\
\years{2011}MIT Robot Locomotion Group representative, Cambridge Science Festival and New Hampshire TechFest\\
%\years{2011}MIT Robot Locomotion Group representative, New Hampshire TechFest\\
\years{2008, 2011}Invited speaker, MIT Women's Technology Program\\
\years{2008}Invited speaker, MIT CSAIL Campus Preview Weekend

\subsection*{Athletic Achievements}
\noindent
\years{2010,2012}US Olympic Wrestling Training Camp participant\hfill Colorado Springs, CO\\
\years{2009-2012}Competitor, regional & national women's tournaments\hfill US \& Canada\\
\years{2010} All-American Wrestler, National Collegiate Wrestling Association\hfill Hampton, VA\\
\years{2008}Team Member, NCAA Div. III Varsity Wrestling Team \hfill MIT

% Wrestler
% • Team Member, MIT’s NCAA Div. III Varsity Wrestling Team Winter ’08 - ’09
% • Competitor, US and Canada in regional & national women’s tournaments ’09 - ’12
% • Two-time Training Camp participant, US Olympic Training Center in Colorago
% Springs, CO Aug. ’10, Sep ’12
% • Board member of the Massachusetts Chapter of USA Wrestling 2012


\section*{References}
\noindent
\begin{multicols}{2}

\textbf{Robert Miller}\\
Professor of Computer Science
\\
\emp{MIT CS \& AI Lab (CSAIL)}
\\\\
\textbf{Bj{\"o}rn Hartmann}\\
Associate Professor of Electrical Engineering \& Computer Science
\\
\emph{University of California, Berkeley}\\\

\textbf{Dan Russell}\\
Senior Research Scientist%, Search Quality \& User Happiness
\\
\emph{Google}\\\\

\columnbreak

\textbf{Scott Klemmer}\\
Professor of Cognitive Science and \\Computer Science \& Engineering\\
\emph{University of California, San Diego}\\

\textbf{Miryung Kim}\\
Associate Professor of Computer Science\\
\emph{University of California, Los Angeles}


\end{multicols}
 

%\hrule








%\vspace{1cm}
\vfill{}
%\hrulefill

\begin{center}
{\scriptsize  Last updated: \today\- •\- 
% ---- PLEASE LEAVE THIS BACKLINK FOR ATTRIBUTION AS PER CC-LICENSE
Typeset in \href{http://nitens.org/taraborelli/cvtex}{
%\fontspec{Times New Roman}
\XeTeX }\\
% ---- FILL IN THE FULL URL TO YOUR CV HERE
\href{http://eglassman.github.io/CV.pdf}{http://eglassman.github.io/CV.pdf}}
\end{center}

\end{document}